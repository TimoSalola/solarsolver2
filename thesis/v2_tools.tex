\chapter{Solar irradiance simulation tools}
Having a mathematical model which would simulate the output of a PV system would allow for the parameters of a PV installation to be solved with model fitting. In the best case scenario, we would have a physics based model which would take geographic location, panel installation angles, time of year and panel surface area or power rating as inputs and the output would be similar to the data from FMI Kumpula installation seen in table \ref{table_fmi_kumpula_csv}. Creating a such model is rather challenging as the model has to take into account atmospheric scattering, Sun angles, Sun-Earth distance variation and a multitude of other factors, the consideration of which are far beyond this mathematics thesis. Luckily the modeling of the energy output of solar PV installations has uses for the cost-benefit analysis of solar PV installations and thus pre-existing modeling algorithms are publicly available. 


% TODO REWRITE

This thesis uses a plane of array irradiance simulation function from the python library PVlib. The function takes geographic location, timestamp and panel angles as inputs. The outputs contain power values which describe the amount of direct and atmospherically scattered light that would hit a square meter sized imaginary solar panel with the input parameters. The sum of these sources is referred as plane of array (POA) irradiance and this value can be used to estimate the output of solar power installations. A section of simulated data is included in table \ref{table_poa_simulated_format}.

As the model simulates radiation values during clear sky conditions and not the power output of pv installations, the model should be seen as an approximation which is accurate to a certain degree. The differences between the model and recorded measurements could be due to reflectivity of the solar panels, weather conditions, temperature related changes in efficiency, atmospheric composition or a multitude of other factors which the model does not take into account.

\begin{table}[h]

\centering

\begin{tabular}{r|cccc} \hline\hline

Timestamp[UTC] & Minute & POA(W) \\ \hline
$2018-05-30$ $00:00$ &  $0$ & $0.0$\\
$2018-05-30$ $00:01$ &  $1$ & $0.0$\\
$2018-05-30$ $00:02$ &  $2$ & $0.0$\\
\vdots & \vdots & \vdots \\
$2018-05-30$ $ 07:34$ & $454$ & $800.691861$\\
$2018-05-30 $ $07:35$ & $455$ & $802.110516$\\
$2018-05-30 $ $07:36$ & $456$ & $803.517424$\\
\vdots & \vdots & \vdots \\
$2018-05-30$ $ 23:57$ & $1437$ & $0.0$\\
$2018-05-30 $ $23:58$ & $1438$ & $0.0$\\
$2018-05-30 $ $23:59$ & $1439$ & $0.0$\\

\hline\hline
\end{tabular}
\tabcaption{One day of simulated plane of array irradiance values. Note that the minute column is added to the table for convinience and it is reduntant as minutes can be read from the timestamps.}
\label{table_poa_simulated_format}
\end{table}

\section{PVlib POA python function} % , parameter space and computational requirements}
Taking a look at the code responsible for the plane of array irradiance simulations can give some insights into the behavior of the estimations and problem of parameter estimation. The  python function responsible for simulating plane of array irradiance values over a day is defined with the header \ref{poa_header}. In this header we can see that the function takes 7 inputs each of which is mandatory, meaning that when the functions is used as a tool for parameter estimation, some of the parameters have to be guessed or randomly assigned.

\begin{lstlisting}[caption={PVlib POA simulation function header.}, label={poa_header}]
def get_irradiance_with_multiplier(year, day, latitude, longitude, tilt, azimuth, multiplier):
\end{lstlisting}







%The following python code is a function which adds the ability to scale the power values of PVlib POA simulations. This ability is needed as by default PVlib simulates the amount of energy radiated towards an imaginary square meter sized solar panel with given coordinates, timestamp and installation angles and as such the power values have to be scaled in order to match the surface areas and efficiency of real installations. The code itself does not tell much of the mathematics used for the simulation, but it can be used to show the input parameters required by the POA simulation function. Each of the 6 parameters is mandatory, meaning that if latitude, longitude, day or any other parameter is left out when the function is called, this will raise an error upon code execution. This means that when the functions is used as a tool for parameter estimation, some of the parameters have to be guessed or randomly assigned. 

%This wrapper shows that six input variables are used for generating a day long POA simulation. The functions requires these parameters, meaning that python will raise an error if the function is called without the necessary 6 inputs. This means that when the function $get\_irradiance\_with\_multiplier()$ is used as a tool to solve the parameters of a system, up to 5 parameters may have to be guessed or assigned random values. The parameters and their ranges are as follows:
\subsection{Function input parameters} 
The following listing contains the parameters of the plane of array irradiance function and their domains.
%The POA simulation function accepts real --or integer as is the case with the day parameter-- valued parameters in the ranges listed below. 

\begin{itemize}
	\item Year $\in \mathbb{N}$
	\item Day [1, 365/366] $\in \mathbb{N}$
	\item Latitude [-90, 90] $\in \mathbb{R}$%, Finland fits within subrange [59, 70]
	\item Longitude [-180, 180]  $\in \mathbb{R}$%, Finland fits within subrange [19, 32] 
  	\item Tilt [0, 90] $\in \mathbb{R}$
  	\item Azimuth [0, 360[ $\in \mathbb{R}$
  	\item Multiplier ]0, $\infty$[ $\in \mathbb{R}$
\end{itemize}

%\noindent \textbf{Note:} While the function does accept the full latitude and longitude ranges as inputs, it may be beneficial to restrict the range of the coordinate parameters when the approximate location of the installation is known. For example, Finland fits within subrange [19, 32] on the longitude axis and thus it could make sense to restrict the longitude range when examining installations located within Finland.

\vspace{3mm}
\noindent The combination of these parameters and their ranges can be thought to form a subspace in seven-dimensional Euclidean space, or six dimensional if year and day are combined into a date variable. This so-called parameter space and its "volume" are both concepts that can be used for analyzing the difficulty of parameter estimation problems, behavior of parameter estimation functions and their efficiency. In general, the more parameters and thus dimensions there are, the larger the resulting parameter space is and the harder the problem becomes. And the more parameters an algorithm is attempting to solve at once, the slower the algorithm can be expected to be.

With solar PV installation parameter estimation, there are 5 unknown parameters as the year and day parameters are always known. If each of the remaining parameters is discretized to 20 evenly spaced values, solving all the 5 parameters at once by testing out every possible combination would require evaluating $20^5$ or 3.2 million unique combinations. However if the parameters could be solved one by one, isolted from the influence of other parameters, there would only be $20*5$ or 100 unique combinations, a reduction of 32000 to 1. This highlights how important it is to break larger problems into smaller problems whenever possible.

%In the worst case where none of the 6 parameters are known and each of the parameter space axes are discretized to $n$ unique evenly spaced values, then the amount of possible combinations is $n^6$. 

%The exponent of 6 may seem insignificant, but even with a crude space discretization where $n$ is chosen to be 20 would result in 64 million combinations. If testing out one parameter combination were to take a second of computing time, testing out each of the 64 million combinations would require more than 700 days of non-stop computing. Luckily the day parameter is always known and thus $n^6$ drops into $n^5$ but even then the size of the parameter space is enormous.

%Thus the goal of the parameter estimation functions should be to reduce the freedom of movement in the parameter space as efficiently as possible, eventually restricting the parameter space into an individual point which corresponds to the installation parameters of the system.


%This means that when unknown parameters of a multi parameter function are being solved, emphasis should be placed on solving the parameters one by one


%In best case scenario, one or more of these parameters can be locked to a specific value. This decreases the volume of the parameter space and the goal 


%the amount of different parameter combinations that may have to be tested while the installation parameters are being solved. 



%The amount of dimensions is relevant as if $n$ discrete values are chosen for each parameter, the total amount of different parameter combinations would be $n^6$. Luckily the day parameter is always known and thus there are only 5 free dimensions, but this still results in a fairly large parameter space. Even a crude discretization with 20 values per dimension would still result in $20^5$ or 3.2 million different parameter combinations. 

%The amount of parameter combinations is relevant as some problems may prove to be difficult to solve intelligently. In these cases, testing out every parameter combination may be the only option. However exhausting the parameter space may not be viable if the space is extremely large.


%With extremely large parameter spaces even this method may become unviable as the computing time and energy required 




%In theory, if we could design a function which could determine if the POA simulation and the real world measurements match, then it would be possible to test out every possible parameter combination until the correct set of parameters was found. In practice however, this would not be adviseable or in some cases even possible as the parameter space is enormous. In the case where the range of each parameter was split into equidistant points, each of which had to be tested for, in total there would be $20^5$ or 3.2 million different sets of parameters. If testing out one set of parameters would take about a second of computing time, then the complete set of 3.2 million unique parameter combinations would require almost 40 days of non stop computations.




%The following python code generates a scaled POA curve for a day of data with given input parameters. As the code is a wrapper for the more complex $get\_irradiance$ fuction, it can be written with just a few lines of code and it should highlight the parameters used by the PVlib simulation function. The 6 necessary parameters are latitude, longitude, day, tilt, facing(azimuth) and multiplier. This means that the parameter space of the POA model is a 6 component vector. Luckily the day can be assumed to be always known and as such there are only 5 unknown parameters which span the parameter space.


%def get_irradiance(lat, lon, day, tilt, facing):

%This thesis uses a plane of array irradiance simulation function from the python library pvlib to simulate the energy output of solar pv installations. The inputs of said function are the geographic location of the installation, timestamp and panel installation angles. And the output of the function is a python pandas dataframe, a table-like entity with timestamps as indices and watts per square meter as the indexed data. 



%According to the PVlib documentation, the plane of array irradiance function returns multiple different power values. Out of these the "poa_global" is a watts per square meter estimation of the direct and indirect diffuse light that would hit a solar panel with the given coordinates, date and panel angles. A such model 

% https://pvlib-python.readthedocs.io/en/v0.9.0/generated/pvlib.irradiance.poa_components.html



%describes the amount of direct and indirect radiation which would hit the surface of a solar panel at given coordiantes, date and panel angles. Note that it does not take into account the reflectivity of the panels, efficiency loss due to heat or a multitude of other factors and the lack of these features will introduce some errors in the results. 
 
% TODO REWRITE END

%In addition, the pvlib poa function included in the appendix \ref{pvlib_poa_code} does not take into account the panel surface area and thus an additional function \ref{pvlib_poa_multiplier_code} was written. And unless otherwise denoted, the POA function \ref{pvlib_poa_code} will be assumed to be a black box function, meaning that the exact behavior of the pvlib function is expected to be unknown.

\newpage
\section{PVlib POA evaluation}

Before using the pvlib POA simulations for parameter estimation, the simulated estimates should first be evaluated. The figure \ref{fig-multidaypoavsmeasurements} indicates that in clear sky conditions the pvlib irradiance model is following real world measurements closely with a few exceptions. During cloudy days, the measurements can be seen to exceed the power generation estimated by the model. The cause for this increase could be reasoned to be the additional sunlight reflected from clouds towards solar panels in partly cloudy weather conditions. Regardless the reason, this shows that the noise introduced in measurements can be positive as well as negative. There is also some deviation visible in the power values of first and last non-zero hours which may prove to be an issue in later stages.

The same figure shows the importance of finding clear sky days as the only charasteristic in the measurement plots which seems to be consistent between the cloud free and cloudy days seems to the timing of the first and last non zero minutes. In the 240th day for example, most of the measurements are either higher or lower than the estimated values, but the first and last non-zero minutes seem to align with the first and last non-zero minutes of POA simulations. On the 70th and 150th day, the difference between the modeled first and last non-zero minutes is visible on the graph. Whether this difference is significant will depend on the algorithms applied.

% Proper evaluation of the irradiance model is not the topic of the thesis and as such 





\begin{figure}%[h]
\centering
\includegraphics[width=1\linewidth]{pics/multiday_vs_neat}
\figcaption{Power output of FMI Kumpula PV installation and the pvlib POA simulation computed with the parameters \ref{table_fmi_helsinki_kuopio_parameters}. Horizontal axis on the graphs corresponds to time and vertical axis marks the estimated power values. The purpose of the graphs is to display the different shapes and deviations from POA models and thus axis names and numbers were left out. Upper row contains randomly selected days while as the lower row has days chosen by a clear sky algorithm mentioned in chapter \ref{clearskyalgo_chapter}. Measurements are from 2017. POA irradiance values were multiplied by 20 in order to match the curves values on power axis.}
\label{fig-multidaypoavsmeasurements}
\end{figure}



% On the 70th and 150th day, the first and last minutes do not seem to be exactly the same as in the simulation, but the difference seems minor and it is occuring in different directions.






%\textit{The following claims are unverified conjectures, but the smooth shape and the early date of the first graph could hint that the increased peak production on the 70th day could be due to reflections from snow, while as the more irregular production on the 150th and 240th day would seem to indicate that the variation is caused by clouds. Filtering out days such as the 150th or the 240th from the dataset should be rather simple as the high frequency component is noticeable, but low frequency deviations such as the smooth increase in production of the 70th day could prove to be more difficult to detect algorithmicly.}






\newpage
\newpage
\newpage
\section{Influence of different parameters on the PVlib poa model}
\label{influence_parameters}
%In order to use the POA model for solving system parameters, each model parameter should 


%the plane of array irradiance function parameters should all have an unique relationship with the 


% the parameters of the POA simulations have to be independent. This means that two different points in the parameter space should not result in the same POA curve as if this occurs, it will not be possible to determine 



%This means that when two different inputs are given to the PVlib plane of array irradiance function, the outputs should never be the same. As 






%In simplified terms, this independence means that no two sets of different parameter should result in the same POA curve, however this property of sameness fairly challenging to measure in a meaningful way due to the complex nature of the POA curves. %Methods for evaluating this difference are still needed as otherwise it would not be possible to mathematically prove that the a POA simulation with certain parameters is a better fit to the measurements than a POA simulation with different parameters.






%The two approaches used in this thesis are significant point differences and area difference. In significant point based difference measurements, a set of sigificant points are chosen from the plots. These could be the peak or last/first minute 


\begin{figure}[ht!]
\centering
\includegraphics[width=0.8\linewidth]{pics/parameter_deltas}
\figcaption{POA irradiance simulations with different parameters. The control contains a curve modeled after FMI Helsinki \ref{table_fmi_helsinki_kuopio_parameters} PV installation and other graphs show how the POA curves change when POA parameters latitude, longitude, day, tilt or facing(azimuth) are adjusted. Adjustments were done in steps of 8 for latitude, longitude 15, day 25, tilt 15 and azimuth 60.}
\label{fig_poa_different_parameters}
\end{figure}

\noindent In the best case scenario each of the simulation function inputs would affect one measureable property in the irradiance plots and their relationship would be bijective. To give an example, if the peak power minute was isolated from all other parameters than the longitude and the relationship between longitude and peak power minute was linear, it would be possible to solve the peak power minute to longitude function with just a few plane of array irradiance simulations.

In the exact opposite case where every measureable property of irradiance plots is affected by every input parameter, solving the parameters would be much harder or even impossible due to limited computational resources. For example if all of the parameters influenced the same traits to different extents and the system was not bijective, multiple parameter combinations could result in the same simulated power graph. In a such system there would not be a single solution but rather a set of possible solutions.

The problem of solving installation parameters lies somewhere in between the two extremes. The longitude parameter would seem to shift the curve along the time axis where as tilt and facing(azimuth) parameters do not affect the first or last non-zero minutes but they do affect the shape of the curve. Observations of parameter to trait interactions are listed on table \ref{table_traits}.



\begin{table}[H]
\centering
\begin{tabular}{r|cc} \hline\hline

 Parameter & Traits affected\\ \hline
 Latitude & Shape, first and last minute times\\
 Longitude & First and last minute times\\
 Tilt & Shape\\
 Azimuth & Shape\\

\hline\hline
\end{tabular}
\tabcaption{Function input to observed trait table.}
\label{table_traits}
\end{table}


%Base on these observations, the relationship between longitude and the First and last minute times would seem like the best starting point for parameter solving.

%If the POA model is assumed to be accurate, the model could be used to simulate the effects of different parameters on power generation. This could provide insights into the relationship between patterns in the data and the parameters of the system. The relevant parameters to simulate and their default values can be seen in table \ref{table_default_parameters_poa_simulations}. In the following simulations, only one of the default parameters is varied. This is done in order to isolate the effect of individual parameters.



%\begin{table}[!ht]
%\centering
%\begin{tabular}{r|c} \hline\hline

% Parameter & Value \\ \hline
% Day & $180$  \\
% Latitude & $60^\circ$  \\
% Longitude & $28^\circ$  \\
% Panel tilt & $30^\circ$ \\
% Panel angle & $180^\circ$  \\
%\hline\hline
%\end{tabular}
%\tabcaption{Default parameters for POA simulation used in this section. }
%\label{table_default_parameters_poa_simulations}
%\end{table}


\newpage

\subsection{Influence of different longitudes}
\label{section_different_longitudes}

\begin{figure}[ht!]
\centering
\includegraphics[width=1\linewidth]{pics/poa_var_lon}
\figcaption{First and last non-zero minutes of each day from year long simulations at different longitudes.}
\label{fig-poa_var_lon2}
\end{figure}

\noindent Based on earlier observations listed in table \ref{table_traits}, solving the longitude of installations would seem like a sensible starting point. The figure comparing the effects of different parameters seemed to suggest that the relationship between longitude and significant minute times is very close to linear and the same is seen here in figure \ref{fig-poa_var_lon2}. In Hagdadi 2017 \cite{navid_australian_article} and in Williams 2012 \cite{older_solar_solver_article} this relationship was used in order to determine the geographic longitude. The algorithms used by both of the articles relies on calculating an approximation for the time of the solar noon based on the average of the first and last minutes, this solar noon minute is then translated into a geographic longitude coordinate.



%There are at least two ways of estimating the longitude from the UTC solar noon time. First method is based on fitting a linear equation to a list of known solar noon to longitude-pairs. This would result in an equation of the form $f(x) = 0.25^\circ* x + b$ where the solar noon minute $x$ is multiplied by the constant $0.25$. The constant of $0.25^\circ$ comes from dividing a full circle by the amount of minutes in a day, 1440. The constant $b$ is around $-180^\circ$ and it is the result of solar noon occuring close to noon.
%In the figure \ref{fig-poa_var_lon2}, the relationship between the first and last minutes of a day and the geographic longitude can be seen to be linear. This linear equation should be of the form $f(x) = 0.25^\circ* x + b$ where the solar noon minute $x$ is multiplied by the constant $0.25$. The constant of$0.25^\circ$ comes from dividing a full circle by the amount of minutes in a day, 1440. The constant $b$ is roughly $-180$ degrees as that is the offset required for adjusting solar noon from



%and the figure \ref{fig-poa_var_lon2} it would seem that the relationship between longitudes and first and last minutes is a good starting point for parameter so


%at least very close to linear. In Hagdadi 2017 \cite{navid_australian_article} and in Williams 2012 \cite{older_solar_solver_article} this relationship was used in order to determine the geographic longitude. The algorithms used by both of the articles relies on calculating an approximation for the time of the solar noon based on the average of the first and last minutes, this solar noon minute is then translated into a geographic longitude coordinate.

% and a similar algorithm is detailed in []..




\newpage

\subsection{Influence of different latitudes}
\label{section_different_latitudes}

\begin{figure}[ht!]
\centering
\includegraphics[width=1\linewidth]{pics/poa_var_lat}
\figcaption{First and last non-zero power minutes of each day from year long simulations at different latitudes}
\label{fig_poa_var_lat}
\end{figure}

\noindent The latitude simulations show that the day length stays fairly consistent for locations close to the equator, but with latitudes of $50^\circ$ and higher, the day to day variation is significant. These POA simulations would imply that the region around equinoxes to be ideal for day length based analysis as there day length is always well defined and the rate of change can be measured. 



%Similar results could be expected for the southern hemisphere. 



% Note that the range of good days is dependent on local weather conditions, geographic location and year to year variation. In the northern Finland it is likely to be shorter and it 


%Thus this thesis proposes linear model fitting around the equinoxes as a method for latitude estimations at high latitudes. Equinoxes are especially suited for this as the rate of change in day lengths should close to its maximum. While the relationship between the rate of change and the latitude does not seem to be linear, the rate should be consistent and independent of the influence of other variables.
